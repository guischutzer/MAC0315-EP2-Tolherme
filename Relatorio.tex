\documentclass[a4paper]{article}

\usepackage[english]{babel}
\usepackage[utf8]{inputenc}

\title{MAC315 - EP2: Implementação da Fase 2 do Método Simplex}

\author{
	Guilherme Schützer - NUSP 8658544 \\
	Tomás Paim         - NUSP 7157602
}

\date{17/05/2015}

\begin{document}
\maketitle

\section{Comentários Iniciais}

Para realizar este exercício, passamos a tratar todos os vetores que vimos em aula como vetores colunas como vetores linha para que o Octave realizasse as operações com naturalidade. Por este motivo, sempre que mencionarmos, por exemplo $c'x$ estamos implementando $cx'$.

\section{Algoritmo ingênuo}


A partir dos argumentos da função \texttt{simplex}, começamos a primeira iteração do algoritmo copiando a matriz $A$ para a $B$ e em seguida varremos o vetor $x$ de trás para frente, para evitar que ao eliminar colunas de $B$ acabemos acessando a posição errada na próxima iteração desse loop, e, para cada elemento $i$, adicionamos ao vetor \texttt{bind} seu índice caso o valor de $x_{i}$ seja diferente de zero (o que implica que aquela é uma variável básica), e ao mesmo tempo construímos o vetor \texttt{cB}, que representa os custos associados às variáveis básicas. Caso contrário, eliminamos a coluna $i$ da matriz $B$, pois a variável $i$ é uma das variáveis não-básicas, já que não há soluções básicas degeneradas. Em seguida, invertemos os vetores \texttt{bind} e \texttt{cB} para que eles estejam de acordo com $x$.
No próximo passo do algoritmo devemos calcular os custos reduzidos associados a cada uma das variáveis não-básicas. Para isso, usaremos a fórmula \begin{math}\bar{c}_{j} = c_{j} - c'_{B}B^-1A_{j}\end{math}. Para isso, usaremos a decomposição $LU$ da matriz $B$.
Assim que encontrarmos pela primeira vez um custo reduzido negativo, tomaremos a variável correspondente a esse custo como $l$, que será a variável que entrará na base. Com $l$ já definida, calculamos o vetor $u = -d_{B}$ e calcularemos $\theta^*$ que corresponde ao máximo que podemos andar na direção $d$ sem violar as restrições, através do método visto em aula que consiste em verificar se $u_{i} > 0$ e $x_{B(i)}/u_{i} < \theta*$, com $\theta*$ começando em $\infty$, para todo $i \in \{B(1), B(2), ..., B(m)\}$. Sempre que isso acontecer, também atualizaremos a variável \texttt{imin} que indica o índice da variável que violará as restrições primeiro, que será a variável a deixar a base.
Uma vez terminado esse processo, terminamos de calcular os $\bar{c}_{j}$ restantes para que possamos imprimí-los.
Em seguida, imprimimos todos os dados relativos a essa iteração da função e atualizamos a matriz $B$ e o vetor $x$ para que possamos passar para a iteração seguinte.



\end{document}

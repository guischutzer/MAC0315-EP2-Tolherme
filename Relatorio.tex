\documentclass[a4paper]{article}

\usepackage[english]{babel}
\usepackage[utf8]{inputenc}

\title{MAC315 - EP2: Implementação da Fase 2 do Método Simplex}

\author{
	Guilherme Schützer - NUSP 8658544 \\
	Tomás Paim         - NUSP 7157602
}

\date{17/05/2015}

\begin{document}
\maketitle

\section{Instruções de execução}

Para extrair e executar o \emph{script} da implementação ingênua da fase 2 do Método Simplex:

\begin{verbatim}
$ unzip EP2-7157602-8658544.zip
$ cd EP2-7157602-8658544/Code
$ octave ep2-opt.m # teste para problema com solução ótima
$ octave ep2-inf.m # teste para problema ilimitado


\end{verbatim}

A saída é do formato especificado no enunciado (mais detalhes na seção 3).

\section{Comentários Iniciais}

\paragraph{}
Para realizar este exercício, passamos a tratar todos os vetores que vimos em aula como vetores colunas como vetores linha para que o Octave realizasse as operações com naturalidade. Por este motivo, sempre que mencionarmos, por exemplo, $c'x$, estamos implementando $cx'$.

\section{Algoritmo ingênuo}

\paragraph{}
A partir dos argumentos da função \texttt{simplex}, começamos a função copiando a matriz $A$ para a $B$ e em seguida varremos o vetor $x$ de trás para frente, para evitar que ao eliminar colunas de $B$ acabemos acessando a posição errada na próxima iteração desse loop, e, para cada elemento $i$, adicionamos ao vetor \texttt{bind} seu índice caso o valor de $x_{i}$ seja diferente de zero (o que implica que aquela é uma variável básica), e ao mesmo tempo construímos o vetor \texttt{cB}, que representa os custos associados às variáveis básicas. Caso contrário, eliminamos a coluna $i$ da matriz $B$, pois a variável $i$ é uma das variáveis não-básicas, já que não há soluções básicas degeneradas. Em seguida, invertemos os vetores \texttt{bind} e \texttt{cB} para que eles estejam de acordo com $x$.\paragraph{}
No próximo passo do algoritmo, vamos chamar a função \verb|simplex_rec|, que é recursiva e recebe como parâmetro a matriz $B$, o vetor \texttt{bind}, o vetor $c_{B}$ e o número da iteração atual, além das variáveis que foram passadas para a função \texttt{simplex}. Dentro da recursão, primeiro devemos calcular os custos reduzidos associados a cada uma das variáveis não-básicas. Para isso, usaremos a fórmula \begin{math}\bar{c}_{j} = c_{j} - c'_{B}B^{-1}A_{j}\end{math}, e para que isso seja possível, usaremos a decomposição $LU$ da matriz $B$, tomando muito cuidado com as transposições.
Assim que encontrarmos pela primeira vez um custo reduzido negativo, tomaremos a variável correspondente a esse custo como $l$, que será a variável que entrará na base. Com $l$ já definida, calculamos o vetor $u = -d_{B}$ e calcularemos $\theta^*$ que corresponde ao máximo que podemos andar na direção $d$ sem violar as restrições, através do método visto em aula que consiste em verificar se $u_{i} > 0$ e $x_{B(i)}/u_{i} < \theta^*$, com $\theta^*$ começando em $\infty$, para todo $i \in \{B(1), B(2), ..., B(m)\}$. Sempre que isso acontecer, também atualizaremos a variável \texttt{imin} que indica o índice da variável que violará as restrições primeiro, que será a variável a deixar a base.\paragraph{}
Uma vez terminado esse processo, terminamos de calcular os $\bar{c}_{j}$ restantes para que possamos imprimi-los.\paragraph{}
Em seguida, imprimimos as variáveis básicas, valor da função objetivo e os custos reduzidos associados às variáveis não básicas relativos a essa iteração da função. Então verificamos se o programa chegou ao fim, isto é, se não há mais nenhuma direção associada a uma variável não-básica na qual o custo diminui, o que significa que encontramos o custo ótimo da função, ou se $\theta^* = \infty$, o que significa que ainda há uma direção na qual o custo diminúi porém podemos ``andar'' infinitamente nesta direção, o que implica que o custo ótimo é $-\infty$.\paragraph{}
Caso não tenhamos chegado no final da recursão continuaremos a imprimir os dados relativos à iteração atual, que são a variável que entra na base, as coordenadas básicas da direção $d$ relativa à variável $l$, $\theta^*$ e a variável que sai da base. Então atualizamos as variáveis que serão mandadas para o próximo passo da recursão: atualizamos o vetor $x$ com o quando ``andamos'' na direção $d$ (ou seja, para cada variável básica $i$, fazemos $x_{B(i)} = x_{B(i)} +\theta^*d_{(B(i)}$) e mudamos $x_{l}$ para $\theta^*$), atualizamos também o vetor \texttt{bind} com a variável que entra na base substituindo a variável que sai, substituimos a coluna associada à variável que saiu da base em $B$ por $A_{j}$, substituimos a componente associada à variável que saiu por $c[l]$ em $c_{B}$ e atualizamos o contador de iterações.\paragraph{}
Por fim, chamamos a função recursiva novamente, com os dados atualizados.

\subsection{Exemplos}

\paragraph{}
Problema com solução ótima:

\begin{verbatim}

$ octave ep2-opt.m

\end{verbatim}

\end{document}

\documentclass[a4paper]{article}

\usepackage[english]{babel}
\usepackage[utf8]{inputenc}

\title{MAC315 - EP2: Implementação da Fase 2 do Método Simplex}

\author{
	Guilherme Schützer - NUSP 8658544 \\
	Tomás Paim         - NUSP 7157602
}

\date{17/05/2015}

\begin{document}
\maketitle

\section{1. Algoritmo ingênuo}
Com as informações recebidas de argumento na função \texttt{simplex}, precisamos da matriz $B$ e seus índices $B(1)...B(m)$. Para tanto, um loop pela solução básica fornecida é suficiente, pois já identificamos os índices que correspondem a um valor não-nulo e criamos a matriz $B$ ao mesmo tempo. Na linguagem, é fácil retirar colunas da matriz, portanto $B$ é facilmente criada ao eliminarmos de $A$ as colunas correspondentes aos índices não-básicos de $x$. $x_{asas}$

$\bar{c_{j}} \bar{c}_{j}$

\end{document}
